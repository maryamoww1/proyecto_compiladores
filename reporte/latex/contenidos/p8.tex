
\section{AFDmin: Minimización del Autómata Finito Determinista}
A partir de esta etapa, las definiciones y las implementaciones de un paso a otro se integraron dentro de un mismo módulo. Esto se hizo con el propósito de reducir la cantidad de archivos en el proyecto.

En esta parte del proyecto se implementó la función \texttt{minimizar}, que toma un AFD completo y devuelve una versión equivalente con el menor número de estados posible. La idea principal fue simplificar el autómata sin cambiar el lenguaje que reconoce.

La implementación se divide en pasos pequeños y claros:
\begin{itemize}
  \item \textbf{Eliminar inalcanzables:} Se quitan los estados a los que nunca se puede llegar desde el inicial. Esto hace que el autómata sea más limpio y fácil de analizar.
  \item \textbf{Completar con nodoMuerte:} Si falta alguna transición, se agrega un estado nodoMuerte que absorbe todos los casos no definidos. Esto asegura que el AFD quede completamente definido.
  \item \textbf{Refinamiento de particiones:} Se agrupan los estados que se comportan igual (tienen las mismas transiciones hacia los mismos tipos de estados). El algoritmo va refinando estos grupos hasta que no haya más cambios.
  \item \textbf{Reconstrucción:} Cada grupo final se convierte en un nuevo estado, generando el AFD minimizado.
\end{itemize}

El resultado es un AFD reducido y estable que servirá como base para los siguientes pasos del proyecto (MDD y lexer). 
