\section{MDD: Construcción de la Máquina Discriminadora a partir de los AFDmin}

Después de minimizar cada autómata con \texttt{AFDMin}, el siguiente paso del proyecto fue construir la \textbf{MDD (Máquina Discriminadora Determinista)}.  
La MDD combina todos los AFDmin en una sola máquina que permite al \textit{lexer} analizar la entrada de y decidir qué token reconocer en cada momento.

Cada AFDmin representa las reglas para un token individual.  
Sin embargo, el \textit{lexer} necesita procesar todo el código fuente en un único recorrido, no correr varios autómatas por separado.  
Por eso, la MDD \textbf{integra todos los AFDmin} y \textbf{discrimina} cuál token corresponde a cada secuencia leída.

\subsection*{Cómo se construye la MDD}
La implementación de \texttt{buildMDD} sigue lo sigunte:
\begin{itemize}
  \item \textbf{Unificación del alfabeto:} Se crea un alfabeto global (\texttt{sigma}) con todos los símbolos que aparecen en los distintos AFDs.  
  \item \textbf{Totalización:} Antes de combinar los AFDmin, cada uno se completa con un estado nodoMuerte para que todas las transiciones estén definidas.  
  Esto evita errores y garantiza que la MDD sea completamente determinista.
  \item \textbf{Estados como vectores:} Cada estado de la MDD representa un \textbf{vector de estados}, uno por cada AFDmin.  
  Por ejemplo, si hay tres AFDs, un estado podría verse como \([q_1, q_2, q_3]\), donde cada componente es el estado actual de un token distinto.
  \item \textbf{Exploración y generación de transiciones:} A partir del vector inicial (formado con los estados iniciales de todos los AFDs), se exploran todas las posibles transiciones para cada símbolo del alfabeto.  
  Si aparece un nuevo vector, se le asigna un número de estado y se sigue expandiendo hasta cubrir todo el espacio alcanzable.
  \item \textbf{Etiquetado de estados finales:} Un estado de la MDD se marca como final si al menos una de sus componentes pertenece a un estado final de su AFD correspondiente.  
  Si hay varios posibles tokens, se aplica una \textbf{prioridad} (por ejemplo, el orden en que se definieron) para decidir cuál token reconocer primero.
\end{itemize}
