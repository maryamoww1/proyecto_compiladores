\section{Lexer: Generación del analizador léxico a partir de la MDD}

Una vez construida la MDD, el siguiente paso fue implementar el \textbf{lexer}, que usa esta máquina para recorrer el texto de entrada y producir los tokens correspondientes.  
El lexer ya no necesita ejecutar varios autómatas por separado, sino que se apoya directamente en la MDD para decidir qué token reconocer en cada punto del código fuente.

\subsection*{Cómo funciona el lexer}
La función principal \texttt{lexerM} toma una MDD y una cadena de entrada, y devuelve una lista de pares \texttt{(token, lexema)}.  
El proceso sigue la idea de recorrer la entrada carácter por carácter mientras se avanza dentro de la MDD:
\begin{itemize}
  \item \textbf{Recorrido con la tabla de transiciones:} Se genera un mapa con las transiciones de la MDD. Por cada símbolo leído, el lexer se mueve al siguiente estado según la tabla.
  \item \textbf{Reconocimiento de tokens:} Si se llega a un estado final, se guarda el lexema reconocido junto con el token que representa (usando las etiquetas definidas en la MDD).
  \item \textbf{Regla de \textit{maximal munch}:} El lexer continúa avanzando mientras existan transiciones válidas y solo detiene el reconocimiento cuando ya no puede seguir.  
  En ese momento, devuelve el último token válido encontrado y reanuda el proceso desde el resto de la cadena.
  \item \textbf{Manejo de errores:} Si se encuentra un símbolo que no pertenece al alfabeto, se lanza un error léxico indicando el carácter inesperado.
\end{itemize}

\subsection*{Construcción del lexer desde expresiones regulares}
La función \texttt{buildLexerFromRegex} hace todo el proceso:  
toma una lista de pares \texttt{(token, expresión regular)} y convierte cada expresión en un AFD minimizado usando las etapas 
Luego, todos esos AFDmin se combinan con \texttt{buildMDD} para formar la MDD global, que finalmente se pasa a \texttt{lexerM}.

De esta forma, el lexer puede reconocer todos los tokens definidos en el lenguaje de forma determinista y con un solo recorrido sobre la entrada.
