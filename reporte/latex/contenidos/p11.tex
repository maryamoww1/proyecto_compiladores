\section{Lenguaje IMP: Definición y lectura}

El lenguaje IMP se define en el archivo \texttt{IMP.md}, donde se describen los tokens del lenguaje mediante expresiones regulares.  
Cada línea asocia un nombre de token con su expresión.

Entre los tokens principales definidos están:
\begin{itemize}
  \item \textbf{num:} reconoce números como \texttt{0}, \texttt{15} o \texttt{-42}.
  \item \textbf{ident:} identifica nombres de variables o funciones.
  \item \textbf{asign:} el operador de asignación \texttt{:=}.
  \item \textbf{opArit:} operadores aritméticos como \texttt{+}, \texttt{-}, \texttt{*}, \texttt{/}.
  \item \textbf{opRel:} operadores relacionales \texttt{<}, \texttt{>}, \texttt{=}.
  \item \textbf{opBool:} operadores booleanos como \texttt{not}, \texttt{and}, \texttt{or}.
  \item \textbf{bool:} los valores lógicos \texttt{true} y \texttt{false}.
  \item \textbf{reservCond, reservCiclo, reservSkip:} las palabras reservadas del lenguaje como \texttt{if}, \texttt{then}, \texttt{else}, \texttt{while}, \texttt{do}, \texttt{for}, \texttt{skip}.
  \item \textbf{puntuacion y delim:} los símbolos \texttt{;}, \texttt{\{ \}}, \texttt{( )}.
  \item \textbf{WS (whitespace):} reconoce espacios, tabulaciones y saltos de línea.  
  Este token no representa un símbolo del lenguaje, pero se incluye para que el analizador léxico pueda avanzar correctamente ignorando los espacios en blanco entre tokens válidos.
\end{itemize}

\subsection*{Lectura y procesamiento del archivo \texttt{IMP.md}}
El \texttt{main} del proyecto inicia leyendo el archivo \texttt{IMP.md} ubicado en la carpeta \texttt{specs/}.  
Primero se eliminan los comentarios con la función \texttt{borraComentarios}, que detecta y borra los bloques marcados con \texttt{*- -*} o \texttt{** **}.  
Luego, \texttt{buildTokensFromSpecs} analiza el texto limpio, extrae los nombres de los tokens (las palabras antes del signo igual \texttt{=}) y construye la lista \texttt{impTokens}, que asocia cada nombre con su expresión regular correspondiente.

\subsection*{Ejecución del programa principal}
El \texttt{main} muestra un pequeño menú con ejemplos de programas IMP para probar el analizador.  
Al seleccionar uno, el programa:
\begin{enumerate}
  \item Lee el archivo de ejemplo y muestra su contenido original.
  \item Quita los comentarios del código fuente.
  \item Analiza cada línea para marcar si tiene o no comentarios.
  \item Se aplica el lexer sobre el código sin comentarios.
  \item Muestra todos los tokens encontrados, incluyendo espacios y símbolos desconocidos.
\end{enumerate}

Y así finalmente simulamos un analizador léxico con nuestro lenguaje IMP :D.